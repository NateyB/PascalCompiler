\documentclass[titlepage]{article}
\usepackage{fancyvrb}

\newcommand{\ShowSourceFile}[2]
{
\VerbatimInput[fontsize=\footnotesize{}, frame=lines,label=#2]{#1}
}
% \RecustomVerbatimCommand{\VerbatimInput}{VerbatimInput}%
% {
% fontsize=\footnotesize{},
%
%  frame=lines,  % top and bottom rule only
%  framesep=2em, % separation between frame and text
%  % rulecolor=\color{Gray},
%
%  % label=\fbox{\color{Black}data.txt},
%  labelposition=topline
% }

\author{Nate Beckemeyer}
\title{\textbf{CS 4013: Compiler Construction: Project 1}}
\date{September 2016}



\begin{document}
    \maketitle
    \section*{Introduction}
    For Project 1, I wrote a lexical analyzer in C for Pascal. The purpose
    of the lexical analyzer is to  break down the Pascal source code into
    the parts needed to construct a parse tree. To achieve that goal, the
    lexical analyzer identifies each lexeme in the code, such as the parts
    of a type declaration, `:' and `integer' or `real', the beginning of a
    program, `program', or perhaps the multiplication operator, `*'. It converts
    this lexeme into a token that later parts of the compiler can readily use.

    The user can specify a source document for the analyzer, and the
    analyzer will create a listing file and a token file. The listing file
    contains the source program, line-by-line, but points out any lexical
    errors that occur. The token file contains the line number, the lexeme
    corresponding to the token, the type of the token, and the attribute
    value of the token.

    \section{Methodology}
    The lexical analyzer is simply a series of machines that parse the file.
    The machines are Whitespace, ID/RES, NumMachine, Grouping, CatchAll,
    RELOP, ADDOP, and MULOP\@. These machines break down the file into its
    corresponding tokens, which are output to a file. A loop repeats this
    process until an end of file token is encountered.

    If errors in the source program are encountered while parsing, the machines
    will send an error. The error will be displayed beneath the corresponding
    line in the listing file and passed as a token to the token file.

    \section{Implementation}
    The reserved words, special to the subset of the Pascal language
    that we're studying, are loaded in from a file at the start of the
    program, along with the proper token category and type, should the
    word be encountered during parsing.

    The Whitespace machine matches all whitespace, and returns the appropriate
    token. This token is usually ignored, except for newline characters, which
    are used to update listing and token file information. The ID/RES machine
    matches reserved words and IDs from the symbol table. If a word that could
    be used as an ID has not been encountered before, it will be added to the
    symbol table. The NumMachine matches numbers: ints, reals, and long reals.
    The Grouping machine matches certain ``grouping'' punctation, such as opening
    and closing parentheses and brackets. The CatchAll machine catches various
    punctuation series. The RELOP machine matches relative (comparative)
    operations, the ADDOP machine matches adding operations (including `or'),
    and the MULOP machine matches multiplicative operations operations
    (including `and').

    The machines are evaluated in the order in which they were specified above.
    The machines are passed a token, a string, and a starting point in that
    string. If the machine matches the token, it updates the token's type,
    attribute, and givens the location of the corresponding lexeme. Regardless,
    it returns the new location of the pointer in the string. That location
    may be the same, if no match were found; however, if a match were discovered,
    this update will signify that a valid token has been generated, and
    there is no need to throw an error.

    If any errors are encountered while parsing, the error will be added
    to a queue. Once the generated token is returned (or an error signifying an
    unrecognized symbol is given), then all of the errors in the queue will
    be pulled and displayed in the listing file, and in the token file.

    \section{Discussion \& Conclusions}
    The analyzer involved learning a lot about C. I'm glad that I took the
    opportunity to try to do this project in this language.

    One aspect of this project that I'm particularly proud of is that it will
    identify all errors in a single lexeme, rather than just one. Ultimately,
    that difference will make life much easier for the programmer.

    I wrote this compiler in C, with no external code of any kind. It was
    compiled with clang on macOS Sierra.

    \clearpage{}
    \section*{Appendix 1: Sample Inputs and Outputs} % Input and output files.
    \ShowSourceFile{../../compiler/reswords.dat}{Reserved Words}
    \subsection{Error-Filled Source File}
    \ShowSourceFile{appendix1/error/errors.pas}{Error Source Code}
    \ShowSourceFile{appendix1/error/listing.txt}{Error Listing File}
    \ShowSourceFile{appendix1/error/tokens.dat}{Error Token File}

    \clearpage{}
    \subsection{Error-Free Source File}
    \ShowSourceFile{appendix1/correct/fib.pas}{Correct Source Code}
    \ShowSourceFile{appendix1/correct/listing.txt}{Correct Listing File}
    \ShowSourceFile{appendix1/correct/tokens.dat}{Correct Token File}

    \clearpage{}
    \section*{Appendix 2: Program Listings}
    \ShowSourceFile{code/dataStructures/linkedList/linkedList.c}{LinkedList.c}
    \ShowSourceFile{code/dataStructures/linkedList/linkedList.h}{LinkedList.h}
    \ShowSourceFile{code/machines/processor.h}{Processor.h}
    \ShowSourceFile{code/machines/processor.c}{Processor.c}
    \ShowSourceFile{code/lexicalanalyzer.h}{LexicalAnalyzer.h}
    \ShowSourceFile{code/lexicalanalyzer.c}{LexicalAnalyzer.c}

\end{document}
